Diagnostic PCR-markers are designed to amplify all members of a target
set of organisms and nothing else. A promising approach to ensure
marker specificity is to compare the target genomes to the genomes of
the closest distinct relatives, the neighbors. Any target region that
distinguishes it from the neighbors stands a good chance of being
unique across the board. The program \ty{fur} implements this idea,
and markers constructed from its output have excellent specificity and
sensifivity~\cite{hau21:fur}.

To use \ty{fur}, users need to know the neighbors of their
targets. But how can neighbors be discovered, if they aren't already
known? To answer this question, consider the toy taxonomy in
Figure~\ref{fig:tax}. The numbers are taxon-IDs, and we assume that we
have a mapping from taxon-IDs to the genome accessions. Let taxa 7 and
4 be our targets. Then we look up the most recent common ancestor of 4
and 7, which is 3. This would imply that 6 also belongs to the target
set. The neighbors are the nodes in the subtree rooted on 3's parent,
that is, taxa 8 through 11. Notice that we look up all nodes in a
subtree, not just the leaves, as we do not want to assume that genome
sequences are only associated with taxa in terminal nodes. In the case
of bacteria, for example, species-level and strain-level taxa can be
associated with genomes.

\begin{figure}
\begin{center}
\pstree[nodesep=2pt,levelsep=0.7cm]{\TR{1}}{
  \pstree{\TR{2}}{
    \TR{8}
    \pstree{\TR{9}}{
      \TR{10}
      \TR{11}
    }
  }
  \pstree{\TR{3}}{
    \TR{\textbf{4}}
    \pstree{\TR{5}}{
      \TR{\textbf{7}}
    }
    \TR{6}
  }
}

\end{center}
\caption{Toy taxonomy.}\label{fig:tax}
\end{figure}

To put this a bit more formally, let $m$ be the most recent common
ancestor of the targets. Their neighbors, $\mathcal{N}$, are then
computed by subtracting the nodes in $m$'s subtree from the nodes in
its parent's subtree
\begin{equation}\label{eq:nei}
\mathcal{N} = s(p(m)) - s(m),
\end{equation}
where $s(v)$ returns the nodes in the subtree rooted on $v$, and
$p(v)$ the parent of $v$.

The package \ty{neighbors} provides programs to support the search for
suitable neighbor genome sequences. It is centered on an \ty{sqlite}
database that combines taxonomy and genome information. The database
is build using the program \ty{makeNeiDb}, which is based on the
packages \ty{tdb}.

Let's call the central database \ty{taxonomyDb}, even though the user
can give it any name. As shown in Figure~\ref{fig:db}, \ty{taxonomyDb}
consists of two tables, \ty{genome} and \ty{taxon}. Each genome comes
from an organism identified by its taxon-ID, has a size, and one or
more replicons, which are lumped in one attribute. Each genome belongs
to a taxon. A taxon has a taxon-ID, a parent in the phylogenetic tree,
and a name.

\begin{figure}
  \begin{center}
    \begin{pspicture}(-1.8,0)(10.9,4)
  \psset{xunit=0.9cm}
  %% Entities
  \rput(1,1){\rnode{genome}{\psframebox{\ty{genome}}}}
  \rput(4.5,1){\dianode{belongs}{\ty{belongs\_to}}}
  \rput(8.0,1){\rnode{taxon}{\psframebox{\ty{taxon}}}}
  \ncline{genome}{belongs}
  \ncline{->}{belongs}{taxon}

  %% Attributes of genome
  \rput(0,0){\ovalnode{replicons}{\ty{replicons}}}
  \rput(-0.5,2){\ovalnode{size}{\ty{size}}}
  \rput(1.5,2){\ovalnode{taxid}{\ty{taxid}}}
  \rput(3.2,0){\ovalnode{accession}{\ty{accession}}}
  \ncline{replicons}{genome}
  \ncline{size}{genome}
  \ncline{taxid}{genome}
  \ncline{accession}{genome}

  %% Attributes of taxon
  \rput(7.7,2){\ovalnode{name}{\ty{name}}}
  \rput(10.0,1.7){\ovalnode{taxid}{\underline{\ty{taxid}}}}
  \rput(10.0,0.5){\ovalnode{parent}{\ty{parent}}}
  \rput(8,0){\ovalnode{rank}{\ty{rank}}}
  \ncline{name}{taxon}
  \ncline{taxid}{taxon}
  \ncline{parent}{taxon}
  \ncline{rank}{taxon}
\end{pspicture}

  \end{center}
  \caption{Diagram of \ty{taxonomyDb}.}\label{fig:db}
\end{figure}

The program \ty{makeNeiDb} constructs this database using the free
relational database management system \ty{sqlite3}. Several Go-drivers
are available for \ty{sqlite3}, I installed mine with the following
command.
\begin{verbatim}
$ go get github.com/mattn/go-sqlite3
\end{verbatim}

The database is built from tables supplied by the NCBI. Two types of
data go into its construction, taxonomy information from the NCBI
taxonomy database and genome lists extracted from GenBank. The NCBI
taxonomy database can be downloaded and unpackged with
\begin{verbatim}
wget ftp.ncbi.nlm.nih.gov/pub/taxonomy/taxdump.tar.gz
tar -xvzf taxdump.tar.gz
\end{verbatim}
The genome lists are divided into prokaryotes, eukaryotes, and
viruses. They can be downloaded with
\begin{verbatim}
wget ftp.ncbi.nlm.nih.gov/genomes/GENOME_REPORTS/prokaryotes.txt
wget ftp.ncbi.nlm.nih.gov/genomes/GENOME_REPORTS/eukaryotes.txt
wget ftp.ncbi.nlm.nih.gov/genomes/GENOME_REPORTS/viruses.txt
\end{verbatim}

Once \ty{taxonomyDb} is constructed, we can query it. For example, we
can count the number of taxa in the database.
\begin{verbatim}
$ sqlite3 taxonomyDb "select count(*) from taxon"
\end{verbatim}

A more pertinent query might be to look up the genomes associated
with \emph{Aerococcus urinae}.
\begin{verbatim}
select genome.taxid, replicons, name 
from genome join taxon 
where name like 'Aerococcus urinae%' 
and genome.taxid=taxon.taxid 
and replicons <> '-'"
\end{verbatim}

The plan is, to write a dedicated program for discovering neighbors
and their genomes from this database. At the moment, this program
doesn't exist yet. Instead, we have the program \ty{nei}, which
discovers neighbors from an index of the NCBI taxonomy database. The
index is computed using \ty{index} based on the package \ty{tax}. The
genomes associated with the neighbors discovered by \ty{nei} can then
be looked up from the database using the script \ty{searchDb.awk}. See
the Tutorial for further details.
